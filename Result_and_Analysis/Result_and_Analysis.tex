\section{Result and Analysis}
\subsection{Parameters}
\begin{table}[h]
    \centering
    \renewcommand{\arraystretch}{1.5} %
    \begin{tabular}{|c|c|}
        \hline
        \textbf{Parameters} & \textbf{Value} \\
        \hline
        \textbf{Total Datasets} & 7  \\
        \hline
        \textbf{Total images} & 446K  \\
        \hline
        \textbf{Trained images (real and fake)} & 203K , 203K \\
        \hline
        \textbf{Tested images (real and fake)} & 10K , 10K \\
        \hline
        \textbf{Validation (real and fake)} & 10K , 10K \\
        \hline
        \textbf{Balanced} &  True\\
        \hline
        \textbf{Epochs} &  10\\
        \hline
        \textbf{Batch Size} &  32\\
        \hline
        \textbf{Image Size} &  256 x 256\\
        \hline
        \textbf{Channels} &  3\\
        \hline
        \textbf{Patches} & 16 x 16\\
        \hline
        \textbf{Encoder Hidden Layers} & 12\\
        \hline
        \textbf{Encoder Layers Dimension} & 768\\
        \hline
        \textbf{MLP size} & 3072\\
        \hline
        \textbf{ Number of Attention Heads } & 12\\
        \hline
        \textbf{Loss Function} & Cross-Entropy Loss \\
        \hline
        \textbf{Normalization} & Layer Normalization \\
        \hline
        \textbf{Activation Function} & GeLU \\
        \hline
        \textbf{Dropout Rate } & 0.2  \\
        \hline
        \textbf{Pooling Strategy } & CLS Token \\
        \hline
    \end{tabular}
    \caption{Model Parameters}
    \label{tab:model-parameters}
\end{table}
\subsection{Evaluation}
The performance of our model is associated with various metrics such as: 
\begin{enumerate}
    \item \textbf{Accuracy:}
           The accuracy metric measures how correctly the model predicts instances by calculating the ratio of correctly classified instances to the total samples.
           \[ Accuracy = \frac{TP + TN}{TP + FP + TN + FN} \]
    
    \item \textbf{Precision:}
           Precision assesses the model's ability to identify positive samples among the actual positives, calculated as the ratio of true positives to the sum of true positives and false positives.

           \[ Precision = \frac{TP}{TP + FP} \]
    
    \item \textbf{Recall (Sensitivity or True Positive Rate):}
           Recall measures the model's ability to precisely identify positive samples from the actual positives, calculated as the ratio of true positives to the sum of true positives and false negatives.

           \[ Recall = \frac{TP}{TP + FN} \]
       
    
    \item \textbf{F1 Score:}
       
           The F1 score, a balance between precision and recall, is advantageous in scenarios with unequal class distribution or equal emphasis on both types of errors. It ranges between 0 and 1, with peak performance at 1.

           \[ F1 = \frac{2 \cdot Precision \cdot Recall}{Precision + Recall} \]
      
    \end{enumerate}
    
% \subsection{Work Completed}
% \subsubsection{User Interface of Mobile application}
% % \begin{itemize}
% %     \item a
% % \end{itemize}
% \subsubsection{Backend}
% % \begin{itemize}
   
% % \end{itemize}
% \subsubsection{Machine Learning Model}
% % \begin{itemize}
    
% % \end{itemize}

% \subsection{Work Remaining}
% % \begin{itemize}
    
% % \end{itemize}

\subsection{UI of Project}
We have used React Native for the development of the mobile application.
Following are the UIs with user authentication, login page and home page where we can upload images to be classified as fake and real.\\

\begin{figure}[ht]
    \centering
    \includegraphics[ height =5in ]{img/loginv3.png}
    \caption{\textit{Login Page }}
\end{figure}

\begin{figure}[ht]
    \centering
    \includegraphics[height= 5in]{img/signup.png}
    \caption{\textit{Sign-up form}}
\end{figure}
\begin{figure}[ht]
    \centering
    \includegraphics[height =5in ]{img/Homepage.png}
    \caption{\textit{Home Page}}
\end{figure}

\begin{figure}[ht]
    \centering
    \includegraphics[height =5in ]{img/uploaderv3.png}
    \caption{\textit{Uploader}}
\end{figure}

\begin{figure}[ht]
    \centering
    \includegraphics[height= 5in]{img/Results.png}
    \caption{\textit{Result}}
\end{figure}

\begin{figure}[ht]
    \centering
    \includegraphics[height= 5in]{img/Historyv2.png}
    \caption{\textit{History page}}
\end{figure}


\begin{figure}[ht]
    \centering
    \includegraphics[height= 5in]{img/profilev2.png}
    \caption{\textit{Profile page}}
\end{figure}