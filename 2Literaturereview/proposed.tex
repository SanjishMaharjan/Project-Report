\subsection{Developed System}

Our Deepfake Detection System, namely DeFaceLab, is a application designed to help individuals identify and tackle the issue of deepfake content. The primary goal is to provide strong detection capabilities for deepfakes. Users can submit images to the system, which utilizes advanced algorithms and machine learning to accurately identify manipulated or fake media. This helps users recognize and address the risks associated with deepfakes, such as spreading false information or violating privacy. The system aims to empower users by giving them the tools they need to protect themselves and others from the harmful effects of encountering deepfakes. With the help of implemented algorithm, the system assists users in detecting and raising awareness about deepfakes, contributing to a safer digital environment.\\
\begin{figure}[h]
    \centering
    \includegraphics[width= 3in ]{img/logoblack.png}
    \caption{DefaceLab}
\end{figure}
\\
In the development of our system, we've incorporated the Vision Transformer (ViT) model, a significant advancement in computer vision. ViT's unique methodology, using self-attention mechanisms, enhances its ability to process entire images simultaneously, making it highly effective in detecting manipulated content. To enhance ViT's accuracy, we've integrated the FaceAlignment library for face detection, ensuring optimal extraction of facial features during the analysis process. This helps users recognize and mitigate the risks associated with deepfakes, such as spreading false information or violating privacy. The system aims to empower users by equipping them with the tools needed to safeguard against the harmful effects of encountering deepfakes.