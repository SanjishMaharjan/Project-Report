\subsection{Algorithm}
The algorithm  for our model is as follows:

\begin{enumerate}
   \item \textbf{Libraries import:}
         \begin{enumerate}
            \item Importing the necessary libraries and modules, including TensorFlow, scikit-learn, and facealignment.
         \end{enumerate}

   \item \textbf{Load and Preprocess Data:}
         \begin{enumerate}
            \item Loading the video data using a video processing library.
            \item Extracting the frames from the video.
            \item Applying the face alignment using the \texttt{facealignment} library to ensure consistent face positions.
            \item Resizing each frame to 256x256 using bilinear interpolation.
         \end{enumerate}

   \item \textbf{Split Data:}
         \begin{enumerate}
            \item Splitting the dataset into training, testing, and validation sets based on the required ratios (used 95-5-5 ratio).
         \end{enumerate}

   \item \textbf{Definition of Model:}
         \begin{enumerate}
            \item Creating a sequential model.
            \item Adding an input layer with shape (256, 256, 3) for a 256x256 image with 3 channels.
            \item Normalizing the pixel values to [0, 1].
            \item Addition of an initial \texttt{Conv2D} layer for channel adjustment.
            \item Implementing ViT-specific preprocessing: \texttt{Conv2D} for linear projection and \texttt{Reshape} for flattening patches.
            \item Adding 12 transformer encoder layers, each comprising multi-head self-attention, dropout (0.2), layer normalization, feedforward neural network with \texttt{Conv2D} and dropout, and layer normalization.
            \item Applying Global Average Pooling.
            \item Adding an MLP Head with a \texttt{Dense} layer for binary classification (real vs. fake).
         \end{enumerate}

   \item \textbf{Compiling the model:}
         \begin{enumerate}
            \item Using the Adam optimizer.
            \item Setting the cross-entropy loss function.
            \item Using accuracy as a metric for evaluation.
         \end{enumerate}

   \item \textbf{Training the model:}
         \begin{enumerate}
            \item Fitting the model to the training data with 10 epochs and a batch size of 32.
            \item Monitoring model performance using the validation set.
         \end{enumerate}

   \item \textbf{Evaluating the Trained Model:}
         \begin{enumerate}
            \item Obtaining the test loss and accuracy by evaluating the model on the test set.
         \end{enumerate}

   \item \textbf{Evaluation Results:}
         \begin{enumerate}
            \item Printing the test loss and accuracy for comprehensive model evaluation.
         \end{enumerate}
\end{enumerate}




