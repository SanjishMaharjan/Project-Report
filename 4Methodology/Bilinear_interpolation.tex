\subsection{Bilinear Interpolation }
Bilinear interpolation is a widely used technique in image processing for estimating pixel values at non-integer coordinates within an image. This method is particularly valuable when scaling or resizing images, offering a smoother transition between neighboring pixels compared to simpler interpolation methods ensuring a smoother and visually appealing transition between pixel values, contributing to improved image quality in various computer graphics and computer vision applications.   
\subsubsection{Steps}
\begin{enumerate}
    \item \textbf{Identify Neighboring Pixels:}Given a non-integer coordinate (x, y), locate the four nearest pixels in the image: (x1, y1), (x1, y2), (x2, y1), and (x2, y2), defining a rectangular region 

    \item \textbf{Calculate Interpolation Weights:} Compute the horizontal (u) and vertical (v) interpolation weights based on the fractional part of the coordinates (x, y).
        \[
    u = x - x1, \quad v = y - y1
    \]

    \item \textbf{Perform Interpolation:} Interpolate along the vertical direction to obtain the interpolated value:
    \[
    I_{\text{interpolated}} = (1 - v) \cdot I_{\text{top}} + v \cdot I_{\text{bottom}}
    \]

    \text{where:}
    \begin{align*}
    I_{\text{top}} & = (1 - u) \cdot I(x1, y1) + u \cdot I(x2, y1) \\
    I_{\text{bottom}} & = (1 - u) \cdot I(x1, y2) + u \cdot I(x2, y2)
    \end{align*}

    \end{enumerate}