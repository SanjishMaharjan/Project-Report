\subsection{Scope}

Our project aims to address the prevalent issue of false videos in the realm of deepfake technology. Due to the limited availability of reliable detection tools, our main focus is on creating an accessible and user-friendly deepfake detection software. The primary objective is to curb the widespread dissemination of deepfakes by offering users a platform where they can easily upload digital content and distinguish between authentic and manipulated materials. Our goal is to contribute to a vigilant digital environment, promoting trust and authenticity in the face of evolving challenges posed by deceptive media content.
\\\\
Our project helps in detecting and addressing deepfake content. However, it's crucial to recognize certain limitations. The effectiveness of our deepfake detection software can be influenced by factors like the size and nature of the input data. We also need to carefully consider the tool's applicability boundaries. The success of the detection process may rely on how we validate the input data. Additionally, we must account for potential dependencies on specific types of input. This report offers a thorough overview of the program, covering details on input size, bounds, validation, dependencies, the input/output state diagram, and key inputs and outputs.
\newpage