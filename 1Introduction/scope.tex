\subsection{Scope}

Our project aims to tackle the current challenge of false video proliferation in the world of deepfake technology. With a scarcity of reliable tools for detecting deceptive content, our focus is on creating a user-friendly deepfake detection software. The main goal is to combat the widespread spread of deepfakes by offering users a platform where they can upload images and distinguish between authentic and deepfake content. Additionally, our project has the potential to expand its reach by developing a browser plugin that automatically identifies deepfakes. This adaptable solution can be implemented across various social media platforms and in different governmental organizations.
\\\\
Our project helps in detecting and addressing deepfake content. However, it's crucial to recognize certain limitations. The effectiveness of our deepfake detection software can be influenced by factors like the size and nature of the input data. We also need to carefully consider the tool's applicability boundaries. The success of the detection process may rely on how we validate the input data. Additionally, we must account for potential dependencies on specific types of input. This report offers a thorough overview of the program, covering details on input size, bounds, validation, dependencies, the input/output state diagram, and key inputs and outputs.
\newpage