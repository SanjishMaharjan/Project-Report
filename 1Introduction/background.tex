\subsection{Background}
At present context of time, the rapid advancements in mobile camera technology and the widespread use of social media platforms have made it easier than ever to create and share digital pictures. Deep learning has played a crucial role in developing technologies that were previously unimaginable. One notable example is modern generative models, which can produce highly realistic images, speech, music, and video. These models have been applied in various fields, such as enhancing accessibility through text-to-speech technology and generating training data for medical imaging.

\noindent The deepfakes are created using deep learning techniques like Generative Adversarial Networks (GANs). These techniques involve two main components: a generator and a discriminator. The generator produces fake content, such as faces or scenes, while the discriminator tries to distinguish between real and fake content. Through an iterative process, the generator improves its ability to create increasingly realistic output, aiming to deceive the discriminator. As training progresses, the generated content becomes more convincing, making it difficult to distinguish whether the content is genuine or artificially created. This technology has both creative potential and serious ethical implications, as it can be used for entertainment purposes but also for generating misleading or harmful content.

\noindent Deep fakes are now widely disseminated on social media platforms, which encourages spamming and the spread of false information. Just picture a deep fake image of Donald Trump getting arrested which was trending on twitter or a deep fake of a well-known celebrity assaulting their supporters.
These types of misinformation can brainwash the audience and are awful and endanger and mislead the general public.

\noindent Deep fake detection plays a crucial part in overcoming such a circumstance. Therefore, we provide a novel deep learning-based method that can successfully separate artificial intelligence-generated fake contents from authentic digital materials. In order to identify deep fakes and stop them from spreading across the internet, it is crucial to develop technology that can detect deepfakes.