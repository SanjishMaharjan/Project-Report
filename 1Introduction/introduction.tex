

\pagenumbering{arabic}
\section{INTRODUCTION}
In the last few years, cybercrime, which accounts for a 67\% increase in the incidents of security breaches, has been one of the most challenging problems that national security systems have had to deal with worldwide [1].
Deepfakes (i.e., realistic-looking fake media that has been generated by deep-learning algorithms) are being widely used to swap faces or objects in video and digital content. This artificial intelligence-synthesized
content can have a significant impact on the determination of legitimacy due to its wide variety of applications and formats that deepfakes present online (i.e., audio, image and video).
Considering the quickness, ease of use, and impacts of social media, persuasive deepfakes can rapidly influence millions of people, destroy the lives of its victims and have a negative impact on society in general [1].
\\ The generation of deepfake media can have a wide range of intentions and motivations, from revenge porn to political fake news..
Deepfakes have also been published to falsify satellite images with non-existent landscape features for malicious purposes [3].
There are numerous captivating applications of deepfakery in video compositing and transfiguration in portraits, especially in identity protection as it can replace faces in photographs with ones from a collection of stock images.
Cyber-attackers, using various strategies other than deepfakery, are always aiming to penetrate identification or authentication systems to gain illegitimate access. Therefore, identifying deepfake media using forensic methods remains
an immense challenge since cyber-attackers always leverage newly published detection methods to immediately incorporate them in the next generation of deepfake generation methods. With the massive usage of the Internet and social media,
and billions of images available on the Internet, there has been an immense loss of trust from social media users. Deepfakes are a significant threat to our society and to digital evidence in courts. Therefore, it is highly important to
obtain state-of-the-art techniques to identify deepfake media under criminal investigation.
As demonstrated in Table 1 (inspired by the figure presented in [1]), tampering of evidence, scams and frauds (i.e., fake news), digital kidnapping associated with ransomware blackmailing, revenge porn and political sabotage are among the
vast majority of types of deepfake activities with the highest level of intention to mislead [1].
