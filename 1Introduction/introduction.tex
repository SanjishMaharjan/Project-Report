

\pagenumbering{arabic}
\section{INTRODUCTION}
Deepfake technology has revolutionized the world of digital media manipulation. By combining artificial intelligence and image/video processing, deepfakes have garnered widespread attention. Deepfakes involve the creation of realistic media portraying individuals in situations they never experienced or saying things they never said. As this technology becomes more sophisticated, concerns arise regarding its impact on politics, entertainment, and personal privacy. This report provides an overview of deepfakes, including their underlying processes, societal implications, ethical challenges and the way of detection. In navigating this landscape, it is crucial to find a balance between innovation and responsible use in our increasingly digitized society.
\\
\\
In the last few years, cybercrime, which accounts for a 67\% increase in the incidents of security breaches, has been one of the most challenging problems that national security systems have had to deal with worldwide.[1]
Deepfakes, at present time, are being widely used to swap faces or objects in video and digital content. This artificial intelligence-synthesized
content can have a significant impact on the determination of legitimacy due to its wide variety of applications and formats that deepfakes present online (i.e., audio, image and video).
Given the speed, simplicity, and effects of social media, convincing deepfakes can easily persuade millions of people, ruin the lives of their victims, and have a detrimental effect, persuasive deepfakes can rapidly influence millions of people, destroy the lives of its victims and have a negative impact on society in general [2].
Deepfake technology has been driven by various motivations, including individual attacks, political manipulation, and the spread of false information. Its impact extends beyond personal attacks to manipulating satellite images and using stock images for identity protection. Cyber attackers continuously adapt their strategies, making it challenging to identify deepfake media and stay ahead of evolving threats.
\\\\
The societal implications of deepfake technology are profound. Misinformation and disinformation fueled by deepfakes erode public trust, damage reputations, and violate privacy at personal and professional levels. Deepfakes can also disrupt democratic processes and contribute to societal polarization. Addressing the legal and ethical concerns surrounding deepfakes requires technological advancements, policy development, media literacy, and careful consideration of privacy rights and the manipulation of visual evidence.[3]
To tackle these implications, it is essential to advance deepfake detection methods, bolster cybersecurity measures, promote media literacy for individuals to discern manipulated content, and establish clear legal frameworks governing the responsible use of deepfake technology.[4] By taking a comprehensive approach, we can effectively navigate the ethical challenges and societal impacts posed by deepfakes.
\\\\
The deepfake technology holds importance in several areas. It offers creative expression and entertainment possibilities, enhances research and development in fields like computer vision, and aids forensic analysis in legal investigations. Deepfakes also emphasize the need for media literacy and critical thinking skills, promoting education and awareness. Ethical considerations and policy development are crucial in addressing the responsible use of deepfakes and protecting individuals' rights. Understanding the significance of deepfake technology enables us to navigate its implications effectively and harness its potential while mitigating potential harm.
\\\\
We cannot dispute the influence deep fakes will have in the next years given all the benefits and cons that have been presented. Therefore, keeping an eye on deepfake content is crucial. This paper will provide an overview of the fundamental organizational structure of our project on how deepfake detections can be done.