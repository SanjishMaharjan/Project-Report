\subsection{Problem Statement}
With the help of visual effects, convincing modifications of digital photographs and videos have been proven for many years. However, new developments in deep learning have dramatically increased the realism of fake content and made it more widely available.   These purportedly artificial intelligence-generated works of media are also known as "deepfakes". It is easy to create deep fakes utilizing artificial intelligence techniques. However, it is extremely difficult to identify these Deep Fakes. In the past, there have been numerous instances of deep fakes being used to effectively incite political unrest, stage terrorist attacks, blackmail individuals, etc. Therefore, it becomes crucial to identify these deep fakes and stop their spread through social media. Therefore, with the growing curiosity we have taken a
step forward in detecting the deep fakes using vggface2 based artificial neural network.