\subsection{Problem Statement}
With the help of visual effects, convincing modifications of digital photographs and videos have been proven for many years. However, new developments in deep learning have dramatically increased the realism of fake content and made it more widely available. These purportedly artificial intelligence-generated works of media are also known as "deepfakes". It is easy to create deep fakes utilizing artificial intelligence techniques. However, it is extremely difficult to identify these Deep Fakes.
Globally, it is found out that about 71\% of total population using internet do not know what a deepfake is. Just under a third of global consumers of internet say they are aware of deepfakes. In the past, there have been numerous instances of deep fakes being used to effectively incite political unrest, stage terrorist attacks, blackmail individuals, etc.
In North America alone, the proportion of deepfakes more than doubled from 2022 to 2023. This proportion jumped from 0.2\% to 2.6\% in the US and it is up from 0.1\% to 4.6\% in Canada and is rapidly growing\hyperref[ref5]{[5]}. Therefore, it becomes crucial to identify these deep fakes and monitor their spread through social media. Hence, with the growing curiosity we have taken a step forward in detecting the deep fakes using different transformer based models.