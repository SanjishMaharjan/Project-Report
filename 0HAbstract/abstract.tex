\begin{center}


    \pagenumbering{Roman}
    \section*{ABSTRACT}
    \justify
    Deepfakes are realistic-looking fake media generated by deep-learning algorithms that iterate through large datasets until
    they have learned how to solve the given problem (i.e., swap faces or objects in video and digital content). The massive generation
    of such content and modification technologies is rapidly affecting the quality of public discourse and the safeguarding of
    human rights. Deepfakes are being widely used as a malicious source of misinformation in court that seek to sway a court’s decision.
    Because digital evidence is critical to the outcome of many legal cases, detecting deepfake media is extremely important and in high demand in digital forensics.
    As such, it is important to identify and build a classifier that can accurately distinguish between authentic and disguised media, especially in facial-recognition systems
    as it can be used in identity protection too. This is what we tend to do. In this work, 
    we will be using the quiet recent Vision Transformer Model, as it uses the attention based mechanism. Since, the Vision transformer has its own unique attributes and is very much reliable and provides a lot of significant merits in development compared to others, we plan to implement it for our work, for the detection of the AI generated images.\\
    \vspace{3 in}
    \\ \textit{Keywords: deepfake detection; digital forensics; media forensics; deep learning; Visual Transformer; AI generated images}


\end{center}