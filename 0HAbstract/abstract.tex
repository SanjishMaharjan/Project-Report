\begin{center}


    \pagenumbering{Roman}
    \section*{ABSTRACT}
    \justify
    Deepfakes are realistic-looking fake media generated by deep-learning algorithms that iterate through large datasets until
    they have learned how to solve the given problem (i.e., swap faces or objects in video and digital content). The massive generation
    of such content and modification technologies is rapidly affecting the quality of public discourse and the safeguarding of
    human rights. Deepfakes are being widely used as a malicious source of misinformation in court that seek to sway a court’s decision.
    Because digital evidence is critical to the outcome of many legal cases, detecting deepfake media is extremely important and in high demand in digital forensics.
    As such, it is important to identify and build a classifier that can accurately distinguish between authentic and disguised media, especially in facial-recognition systems
    as it can be used in identity protection too. In this work, we compare the most common, state-of-the-art face-detection classifiers such as Custom CNN, VGGface2, and DenseNet-121
    using an augmented real and fake face-detection dataset. Data augmentation is used to boost performance and reduce computational resources. Our preliminary results indicate that VGG19 has the best performance
    and highest accuracy of 95\% when compared with other analyzed models.\\
    \vspace{3 in}
    \\ \textit{Keywords: deepfake detection; digital forensics; media forensics; deep learning; VGGface2; face-image manipulation}


\end{center}